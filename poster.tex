% NEEDS TO BE COMPILED WITH pdflatex

\documentclass[portrait,a0paper]{baposter}

\usepackage{graphicx}
\usepackage[utf8]{inputenc}
\usepackage{tabto}
\usepackage{booktabs}
\usepackage{amsmath}
\usepackage{float}
\usepackage{multirow}
\usepackage[hyphens]{url}
\usepackage{enumitem}
\usepackage{ragged2e}
\usepackage{qrcode}
\usepackage{textcomp}

\newlength{\mytextsize}
\setlength{\unitlength}{1.0cm}

%###########################################################################################################################
% Customize here ############################################################################################################

\def\posterTitle{%
    Data-driven low-code programming system
}
\def\posterAuthor{%
   Jaroslav Švarc 
}
\def\posterSchool{%
  Faculty of Mathematics and Physics, Charles University
    }
\def\posterYear{%
    2025
}
\def\posterLogoFaculty{logos/mff-white.pdf}
\def\posterLogoSchool{logos/uk-white.pdf}

\definecolor{maincolor}{RGB}{119, 221, 119}
\definecolor{headercolor}{RGB}{255,255,255}
\definecolor{textbackgroundcolor}{RGB}{255,255,255}
\definecolor{highlightcolor}{RGB}{70,150,255}
\colorlet{highlightbackgroundcolorOne}{red!60!orange!03}
\colorlet{highlightbackgroundcolorTwo}{red!60!orange!15}
\def\backgroundcolor{gray!25}

\usepackage[scaled]{helvet}
% Uncomment for non-serif text font
% \renewcommand*\familydefault{\sfdefault}

%###########################################################################################################################
% My own macros and commands ###############################################################################################

% Custom red markings for TODO items
\newcommand{\todo}[1]{%
    \textcolor{red}{\textbf{#1}}% Bold and red text
    \PackageWarning{TODO}{#1}% Warning in LaTeX log
}
% Custom headerbox command to left-align text
\newcommand{\leftalignedheaderbox}[3]{\headerbox{#1}{#2}{%
    % Uncomment for left-aligned text
    % \RaggedRight{%
    #3
    % }
}}
% Custom headerbox command with highlighted background
\newcommand{\highlightheaderbox}[3]{\leftalignedheaderbox{#1}{#2,%
boxColorOne=highlightbackgroundcolorOne,boxColorTwo=highlightbackgroundcolorTwo%
}{#3}}
% Custom headerbox without a header, intended for additional info at the bottom of the poster
\newcommand{\footerbox}[2]{\headerbox{}{#1,%
above=bottom,borderColor=maincolor,boxheaderheight=1pt%
}{#2}}
% Text highlighting
\newcommand{\highlight}[1]{\textbf{\color{highlightcolor}#1}}

%###########################################################################################################################
\begin{document}
\begin{poster}
	{
		grid=false, % Debugging or Layour grid
		eyecatcher=false, % Custom main header
		columns=6, % for flexibility -- changing 2/3 columns with columnspan
		%
		background=plain,
		bgColorOne=\backgroundcolor,
		%
		headerfont=\bf\Large\sffamily,
		headerFontColor=headercolor,
		headershape=rectangle,
		headershade=plain,
		headerColorOne=maincolor,
		headerborder=open,
		%
		boxshade=shadetb,
		boxColorOne=textbackgroundcolor,
		boxColorTwo=textbackgroundcolor,
		textborder=rectangle,
		borderColor=maincolor,
		boxpadding=1em
	}{}{
		% poster title
		\hspace*{-0.5mm}
		\begin{picture}(23.7, 3)
			\fboxsep0pt
			\put(-0.196, -0.6){\colorbox{maincolor}{\rule[96pt]{675.82pt}{0pt}}}
			\thicklines
			% minipage box for title and authors
			% Faculty logo
			\put(0.24,-0.24){\includegraphics[height=75pt]{\posterLogoFaculty}}
			\put(2.55, 2.35){
				\begin{minipage}[t][96pt]{0.75\textwidth}
					\begin{center}
						{\huge\bf\color{headercolor}\sffamily%
							\posterTitle \\[0.35cm]
						}
						{\large\color{headercolor}\sffamily%
							\posterAuthor
							\raisebox{0.1em}{$\bigm\lvert$}
							\posterSchool
							\raisebox{0.1em}{$\bigm\lvert$}
							\posterYear
							\\
						}
					\end{center}
				\end{minipage}}
			\put(20.52,-0.24){\includegraphics[height=75pt]{\posterLogoSchool}}
		\end{picture}
	}{}{}
	% ==========================================================================================================================
	% Footer ===================================================================================================================
	% Puting the footer first might seem counter-intuitive but trust me on this one.
	% In order to reference the footer elements for bottom-alignments, the footer boxes need to be defined first.
	\footerbox{%
		name=repository,column=2,span=4%
	}{
		\begin{minipage}{.17\textwidth}
			\begin{center}
				\qrcode[height=2cm]{https://github.com/JerrySvarc/InterfaceSmith}%
			\end{center}
		\end{minipage}%
		\begin{minipage}{.66\textwidth}
			\textbf{\large\sffamily\color{maincolor}Repository}
			\\[0.25em]
			\scriptsize\url{https://github.com/JerrySvarc/InterfaceSmith}
			% \vfill\hrule\vfill
			\vspace{0.55em}\hrule\vspace{0.55em}
			\hfill\textbf{\large\sffamily\color{maincolor}Thesis}
			\\[0.25em]
			\scriptsize\url{https://jerrysvarc.github.io/bachelor-thesis/thesis.pdf}
		\end{minipage}%
		\begin{minipage}{.17\textwidth}
			\begin{center}
				% If the thesis URL is short, I would right align it also: BTW I have no idead why the \phantom{a} helps,
				% but it helped me, so might aswell leave it for other unfortunate souls
				% \phantom{a}\hfill%
				\qrcode[height=2cm]{https://jerrysvarc.github.io/bachelor-thesis/thesis.pdf}%
			\end{center}
		\end{minipage}%
	}
	\footerbox{%
	name=supervisor,column=0,span=2,aligned=repository% Although this box is to the left of the aligning box,
	% and thus would be logical to define it first, the QR codes were messy when I worked on this template
	% so I leave the QR codes box to choose its height and then alighn this one next to it.
	}{
	% By all means, laugh away at the brutal by-hand vertical spacing...
	% But then also feel free to change it.
	\textbf{\sffamily\large\color{maincolor}Supervisor}
	\\[0.4em]
	{\large Mgr. Tomáš Petříček, Ph.D.}
	\\[0.1em]
	Department of Distributed and Dependable Systems
	}
	% ==========================================================================================================================
	\leftalignedheaderbox{Intro}{%
		name=intro,column=0,row=0,span=3% Note the `name` value, used later for layout definition
	}{
		Low-code programming systems provide a graphical user interface (GUI), through which users can create software elements.
		Software development using low-code programming systems is increasingly popular and
		traditional low-code development systems that provide user interface creation functionality primarily provide the \emph{UI-to-data} approach, where developers create user interface elements before populating them with data.
		However, the data-to-UI approach, where the development process begins with concrete data that drives the creation of corresponding UI elements, remains unexplored as a primary development method.
		We present the InterfaceSmith prototype programming system, which implements \emph{data-to-UI} as the primary development method for creating web applications' UI elements.
		The system aims to aid developers in modifying the interface through context menus and generates applications following the Elm architecture. Our evaluation through benchmarks, including a TO-DO list application and tasks from the 7GUIs benchmark suite,
		demonstrates the system's effectiveness in reducing the amount of code developers need to write while maintaining the ability to implement custom web application functionality.
	}

	\leftalignedheaderbox{Motivation}{
		name=motivation,column=0,span=3,below=intro% Note how `aligned` defines that this box starts exactly with the specified box
	}{
		The primary motivation for this research is to allow the creation of single-page web applications following the Elm architecture, also known as Model-View-Update, based on concrete
		data uploaded to the system. The aim is to allow incremental creation of UI elements based on the uploaded data's type and structure.
	}

	% --------------------------------------------------------------------------------------------------------------------------
	\leftalignedheaderbox{Goals}{
		name=goals,column=0,span=3,below=motivation% Note how `below` defines that this box is exactly below the intro box
	}{
		\begin{enumerate} [leftmargin=*]
			\item Explore the
			\item Create a working \textbf{prototype programming system} implementing the data-driven approach.
			\item Benchmark the prototype application on the following tasks:
			      \begin{itemize}
				      \item A simple \textbf{TO-DO list} application inspired by the \emph{TodoMVC} becnhmark.
				      \item \textbf{Counter} task from the \emph{7GUIs} becnhmark.
				      \item \textbf{Temperature converter} task from the \emph{7GUIs} becnhmark.
			      \end{itemize}
		\end{enumerate}
	}
	% --------------------------------------------------------------------------------------------------------------------------

	% --------------------------------------------------------------------------------------------------------------------------
	\leftalignedheaderbox{Solution approach}{% Note how this box is highlighted
		name=solutionApproach,column=3,span=3
		% Note how `row` defines an arbitrary vertical placement in the layout
		% - see the `grid` option of the `poster` environment to enable the layout grid
		% Note how the combination of the `column` and `span` options places the box at a central position
		% However, for a "central result box" layout, I suggest changing the `columns` option of the `poster` environment
		% to a different value, allowing for better ratios between spanning columns.
	}{

	}
	% --------------------------------------------------------------------------------------------------------------------------
	\leftalignedheaderbox{Experiments}{%
		name=ssira,column=3,span=3,below=solutionApproach % Note how `above` defines that this box ends exactly above the specified box
	}{

	}
	% --------------------------------------------------------------------------------------------------------------------------
	\leftalignedheaderbox{Summary}{%
		name=summary,column=3,span=3,above=repository, below=ssira % Note how `bottomaligned` defines that this box ends exactly where the specified box ends
	}{


	}
\end{poster}
\end{document}
